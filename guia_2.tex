% preambulo
\documentclass[11pt,a4paper]{article}

\usepackage[utf8]{inputenc}
\usepackage{amsmath}
\usepackage{amsfonts}
\usepackage{amssymb}
\usepackage[margin=1in, footskip=0.25in]{geometry} % para configurar los margenes
\usepackage[shortlabels]{enumitem} % para usar enumerate con letras
\usepackage{commath} % para usar \norm{} 
\usepackage{mathabx} % para usar \check{} para versores
\usepackage{siunitx} % para usar \ang{}
\usepackage{stmaryrd} % para usar \sslash

\title{\'Algebra y Geometr\'ia Anal\'itica} % crea el titulo
\date{}

\begin{document} % inicia el documento

\maketitle % imprime el titulo

\begin{enumerate}
\setcounter{enumi}{0} % inicia el contador

% Ejercicio 1
\item Hallar las ecuaciones de la recta en su forma sim\'etrica, param\'etrica vectorial y param\'etrica escalar para los siguientes elementos geom\'etricos dados:

\begin{enumerate}[a)]
\item Los puntos $P_{1} = (-1, 3)$ y $S = (2, 5)$
\item El punto $P_{2} = (-1, 4)$ y su direcci\'on paralela al vector $\vec{d} = (4, -1)$
\end{enumerate}

% Ejercicio 2
\item Hallar las ecuaciones de la recta en su forma sim\'etrica, param\'etrica vectorial y param\'etrica escalar para los siguientes elementos geom\'etricos dados:

\begin{enumerate}[a)]
\item Los puntos $P_{1} = (-3, 2, 1)$ y $P_{2} = (1, 5, 4)$
\item Los puntos $Q_{1} = (1, 1, 0)$ y $Q_{2} = (1, -3, 2)$
\item El punto $P_{2} = (1, 4, 2)$ y su direcci\'on paralela al vector \linebreak $\vec{d} = (-2, 2, 3)$
\item El punto $A_{1} = (3, -2, 1)$ y el vector director $\vec{s} = (5, 0, 2)$
\item El punto $B_{1} = (3, 2, 4)$ y con direcci\'on perpendicular al vector $\vec{d} = \check{i} + 2 \check{j} + \check{k}$
\end{enumerate}

% Ejercicio 3
\item Haciendo uso de GeoGebra resuelva los puntos 1) y 2) anteriores.

% Ejercicio 4
\item Encontrar puntos que pertenezcan a las siguientes rectas:

\begin{enumerate}[a)]
\item $(x, y, z) = (1, -3, 2) + k (1, 1, 4)$
\item $\begin{cases}
  x = 1 + 2k \\
  y = -2 + 3k \\
  z = 4 + 2k
\end{cases}$
\item $\frac{x + 1}{3} = y + 1 = z$
\item $\frac{x + 2}{4} = y - 2 = z + 3$
\end{enumerate}

% Ejercicio 5
\item Dados los siguientes puntos $P = (1, -1, 2)$ y $Q = (0, 2, 3)$. ¿A cu\'al o cu\'ales de las siguientes rectas pertenecen?

\begin{enumerate}[a)]
\item $(x, y, z) = (2, 3, 4) + k (1, -4, 2)$
\item $\begin{cases}
  x = 1 + k \\
  y = 4 -2 k \\
  z = 3 + k
\end{cases}$
\item $\frac{x + 1}{2} = \frac{y + 3}{4} = \frac{z - 1}{2}$
\end{enumerate}

% Ejercicio 6
\item Haciendo uso de GeoGebra resuelva el punto 5).

% Ejercicio 7
\item Hallar la ecuaci\'on normal y general del plano dados los siguientes elementos:

\begin{enumerate}[a)]
\item El punto $P_{1} = (1, 1, 3)$ y el vector normal $\vec{n} = (-1, 4, 3)$
\item El punto $P_{2} = (-1, 4, 5)$ y el vector normal $\vec{n} = 2 \check{i} + \check{j}$
\end{enumerate}

% Ejercicio 8
\item Encontrar la ecuaci\'on param\'etrica vectorial y escalar del plano que pasa por tres puntos:

\begin{enumerate}[a)]
\item $P_{1} = (1, 1, 2)$, $P_{2} = (5, 4, 3)$ y $P_{3} = (4, 2, 1)$
\end{enumerate}

% Ejercicio 9
\item Encontrar la ecuaci\'on normal del plano a partir de los puntos $P_{1} = (1, 1, 2)$, $P_{2} = (-1, 0, 5)$ y $P_{3} = (2, 6, -1)$

% Ejercicio 10
\item Dados los siguientes puntos $P = (1, -1, 2)$ y $Q = (3, 2, 5)$. ¿A cu\'al o cu\'ales de los siguientes planos pertenecen? 

\begin{enumerate}[a)]
\item $(x - 2, y + 3, z - 1) * (1, 4, 2) = 0$
\item $2x - 3y + 5z = 2$
\item $x + 3y = 0$
\end{enumerate}

% Ejercicio 11
\item Dados los siguientes planos encontrar puntos que pertenezcan a ellos:

\begin{enumerate}[a)]
\item $2x + y - 5z = 4$
\item $-y + z = 2$
\item $x + 3z = 0$
\item $(x - 1, y + 3, z - 2) * (-1, 0 , 5) = 0$
\end{enumerate}

% Ejercicio 12
\item Dado el plano $-x + y = 0$ y la recta $(x, y, z) = (1, -1, 0) + k (2, 1, 3)$ indicar si son:

\begin{enumerate}[a)]
\item Paralelos
\item Perpendiculares
\item Ni paralelos ni perpendiculares
\end{enumerate}

% Ejercicio 13
\item Dados los siguientes planos $2x + 3y - z = 1$ y $4z = -1$, indicar si son:

\begin{enumerate}[a)]
\item Paralelos
\item Perpendiculares
\item Ni paralelos ni perpendiculares
\end{enumerate}

% Ejercicio 14
\item Dados los siguientes planos encontrar el \'angulo entre ellos:

\begin{enumerate}[a)]
\item $\begin{cases}
  \pi \colon x - y + 2z = 2 \\
  \alpha \colon -x - 2y + 3z = 1
\end{cases}$
\item $\begin{cases}
  \pi \colon x + 2y  = 0 \\
  \alpha \colon -3y + 4z = 0
\end{cases}$  
\item $\begin{cases}
  \pi \colon x = 0 \\
  \alpha \colon y - 2z = 1
\end{cases}$
\end{enumerate}

% Ejercicio 15
\item Dadas las siguientes rectas encontrar el \'angulo entre ellas:

\begin{enumerate}[a)]
\item $\begin{cases}
  (x, y, z) = (1, -1, 3) + k (2, 0, -1) \\
  \frac{x + 3}{2} = y = \frac{z - 1}{5} 
\end{cases}$
\item $\begin{cases}
 \begin{cases}
  x = 1 + 3k \\
  y = k \\
  z = 2 + 2k
 \end{cases} \\
  \frac{x - 1}{2} = 2y = z + 1 
\end{cases}$
\end{enumerate}

% Ejercicio 16
\item Dados los siguientes planos y rectas encontrar el \'angulo entre ellos:

\begin{enumerate}[a)]
\item $\begin{cases}
  -2x + y - z = 2 \\
  (x, y, z) = (1, 1, 5) + k (1, 3, 1)
\end{cases}$  
\item $\begin{cases}
  x - 2y + 3z = 0 \\
  x + 1 = \frac{y - 3}{8} = z
\end{cases}$
\end{enumerate}

% Ejercicio 17
\item Encontrar la distancia del punto $P = (2, 5, 3)$ al plano $-2x + y + 3z = 0$

% Ejercicio 18
\item Encontrar la distancia entre la recta $(x, y, z) = (1, 1, 5) + k (-2, 1, 3)$ y el punto $P = (4, 3, 2)$

% Ejercicio 19
\item Encontrar las distancias entre los siguientes entes geom\'etricos:

\begin{enumerate}[a)]
\item El plano $-3x + 2y + z = 1$ y la recta ($x, y, z) = (1, -5, 1) + k (2, 2, 2)$
\item El plano $-x + 2y - z = 4$ y el plano $-2x + 4y - 2z = 0$
\item La recta $(x, y, z) = (3, -2, 1) + k (2, 1, 3)$ y la recta $\frac{x - 2}{4} = \frac{y - 1}{2} = \frac{z}{6}$
\item El plano $-2x + y = 0$ y el plano $-4x + 3y = 1$
\end{enumerate}

% Ejercicio 20
\item Debate con tus compa\~neros sobre la veracidad o falsedad de las si\-guien\-tes afirmaciones. Para que tu respuesta sea v\'alida debes justificarla adecuadamente, por lo cual puedes emplear definiciones, ejemplos, gr\'aficos, c\'alculos, etc.

\begin{enumerate}[a)]
\item El vector director de una recta (de $\mathbb{R}^{2}$ o $\mathbb{R}^{3}$ ) que pasa por dos puntos conocidos $A$ y $B$, puede hallarse como $A-B$.
\item Si $(x, y, z) = (1, x + 2, -y)$ entonces $x = 1, y = 3$ y $z = -3$
\item El punto $P = (5, 3, 1)$ no pertenece a la recta $(x, y, z) = (3, 1,-2) + k (2, 1, 3)$
\item Las rectas $y = 3x - 2$ y $(x, y) = (1, 2) + k (-1, 3)$ son paralelas.
\item Las rectas $\frac{(x - 1)}{3} = \frac{(y - 1)}{2} = \frac{(z - 1)}{1}$ y $(x, y, z) = (1, 1, 1) + k (-1, 0, 3)$ son perpendiculares.
\item Siempre puede escribirse la ecuación sim\'etrica de la recta (en $\mathbb{R}^{2}$ o $\mathbb{R}^{3}$ ).
\item La recta $\frac{(x + 3)}{3} = \frac{(y - 2)}{2} = \frac{(z - 1)}{1}$ corta al plano $yz$ en el punto $(0, 2, 4)$, al plano $xz$ en $(-6, 0, 0)$ y al plano $xy$ en $(0, 0,1)$.
\item Un plano queda completamente determinado por dos de sus puntos y un vector paralelo a \'el.
\item Un plano queda completamente determinado por tres cualesquiera de sus puntos.
\item Un plano queda completamente determinado por uno de sus puntos y un vector perpendicular a \'el.
\item Si nos dan las ecuaciones de dos rectas que se cortan, podemos hallar sin inconvenientes la ecuaci\'on del plano que determinan.
\item Si nos dan las ecuaciones de dos rectas paralelas, tenemos elementos suficientes como para hallar las ecuaciones del plano que determinan.
\item Si nos dan las ecuaciones de dos rectas cualesquiera en el espacio, podemos hallar sin inconvenientes la ecuaci\'on del plano que determinan.
\item Un plano es paralelo a una recta cuando cualquier vector normal al plano es paralelo al vector director de la recta.
\item Dos planos paralelos tienen vectores normales que son uno m\'ultiplo escalar del otro.
\item Dos planos perpendiculares tienen vectores normales paralelos.
\item Una recta es perpendicular a un plano cuando su vector director es paralelo al plano.
\item Si una recta es perpendicular a un plano, entonces cualquier vector paralelo a la recta es perpendicular a cualquier vector paralelo al plano.
\item Dos vectores no colineales determinan una familia de planos.
\item Dos vectores cualesquiera determinan una familia de planos.
\item El \'angulo entre dos rectas puede calcularse mediante el \'angulo entre los vectores directores de las rectas.
\end{enumerate}

\end{enumerate} % cierra la enumeracion de los ejercicios

\end{document} % cierra el documento
